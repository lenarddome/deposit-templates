\documentclass[man]{apa6} % options are man for manuscript; jou for journal article; or doc for document
% Uncomment these lines for Ariel fonts
% \usepackage{fontspec}
% \setmainfont{Arial}
\usepackage{url}
\usepackage{apacite}
\bibliographystyle{apacite}
% header for doc format
% \lhead{10557964}
% \rhead{My Apa stuff \thepage}
% R code blocks
% have to find one form python and c++ as well
\let\proglang=\textsf
\newcommand{\R}{\proglang{R}}
\newcommand{\pkg}[1]{{\normalfont\fontseries{b}\selectfont #1}}
\newcommand{\Rfunction}[1]{{\texttt{#1}}}
\newcommand{\fun}[1]{{\texttt{#1}}}
\newcommand{\Robject}[1]{{\texttt{#1}}}

\title{My APA template for Plymouth University}
\author{10557964}
\affiliation{School of Psychology \\ Plymouth University}

\abstract{This is my template that can be used for assignments. It is not done...obiously. I still need to learn how to use zotero and latex. I should also figure out how to get some basic spellchecking stuff done in the editor. Maybe I should do some online course on lynda.com on how to review your grammar. Though, my grammar is quite alright.}

\shorttitle{APA style manuscript}
% in doc class it throws you an error message
\rightheader{APA style manuscript}
\leftheader{NAME}

\begin{document}
\maketitle

Here, I can write my introduction. A long empirical review in a literature coupled with identifying the gap that I am filling with my research.

Is it going to be a new paragraph?
Whitespace signals a new paragraph. Make sure, that line-wrapping is enabled. But,
it is more important, that you make the source-code readable.

\emph{It is very important to compare the `manuscript' version of this document with the final `journal' view.  Using \LaTeX we can go back and forth between these two formats with ease.  Tables and Figures need to appear at the end of the manuscript version, even though they appear embedded in the middle of the printed version.  }

But you also have to put in some source code, because you did some simulation and thought: 'I might explain what is happening here'.

\section{Experiment 1}

Here is the first section. It can be one of the experiments you have done or the Method or anything.
\subsection{Method}

This is how a Subsection looks like.
The ``method'' is a subsection of the experimental presentation in which all the details of setting up and conducting the experiment are described.  There are a number of more or less standard components to a method, shown below.  %from AP

%
\subsubsection{Participants}
Lab animals.
\subsubsection{Apparatus}
Laser pointer.
\subsubsection{Procedure}
I have don this to them.

\subsection{Results}

Here you show what your results are.

\begin{table}[tbp]
\caption{Some numbers that could be experimental data.}
\label{tab:tab2}
\begin{tabular}{lcc}\hline
          & \multicolumn{2}{c}{Factor 2} \\ \cline{2-3}
Factor 1  & Condition A  & Condition B   \\ \hline
First     & 586 (231)    & 649 (255)     \\
          &    2.2       &    7.5        \\
Second    & 590 (195)    & 623 (231)     \\
          &    2.8       &    2.5        \\ \hline
\end{tabular}
\end{table}


\begin{figure*}[htbp]
\begin{center}
\caption{Alternatively, the figure can be made a large figure by using the figure* command. This figure was generated using the R statistics and graphics program.  This is a demonstration of a `notched boxplot' for ten samples taken from a normal distribution, and ten taken from a mixture of two normals with different variances. }
\label{fig:bigfig}
\end{center}
\end{figure*}


\subsection{Discussion}

This is a specific discussion relevant to the outcome of experiment 1.

\section{General Discussion}
This is a discussion on what your results imply, what is its place in the bigger picture and what is next.%from AP

\subparagraph{Bibliography stuff}
It is important to note that books \cite{leary} are cited in the bibliography differently than chapters \cite{rev:ea07} or articles \cite{killeen}.  The specific rules for citations are complicated, but actually  are included in such bibliographic tools as the open source format BibTex (which can be used by \LaTeX using the  apa.cite commands) or the commercial program EndNote.  For Northwestern students, EndNote may be downloaded for personal use.  The official guide to how to include references is the APA style manual \cite{apa:6}, but ``cheat sheets" may be found on the web by searching for `APA style'.

The final paragraph of the discussion section is the \emph{take home message}.  Tell us once again how and why your results are important.

% have to make a bibetex file with the references you use
% in this case it is test2.bib
\bibliography{test2}

\end{document}
